\documentclass[notitlepage,11pt,a4paper,final,twoside]{article}
\usepackage[utf8]{inputenc}
\usepackage[spanish]{babel}
\usepackage[pdftex]{graphicx}
\usepackage[scale=0.7,marginparwidth=1.2in]{geometry}
\usepackage{booktabs}
\usepackage{rail}
\usepackage{color}
\usepackage{paralist}

%======================================================================
% Font and font encoding
%----------------------------------------------------------------------
\usepackage[T1]{fontenc}           % accentuated letters in pdf fonts
\usepackage{lmodern}               % replacement for computer modern
\usepackage[scaled=.8]{beramono}   % beramono monospaced font
\usepackage{scalefnt}              % Allows font escalation
%======================================================================


%======================================================================
% Colors
%----------------------------------------------------------------------
% From ``named colors'' palette:
\definecolor{lightcyan1}{rgb}{0.88, 1   , 1}    % 224, 255, 255
\definecolor{lightcyan2}{rgb}{0.82, 0.93, 0.93} % 209, 238, 238
\definecolor{lightcyan3}{rgb}{0.7,  0.80, 0.80} % 180, 205, 205
\definecolor{lightcyan4}{rgb}{0.48, 0.55, 0.55} % 122, 139, 139
\definecolor{lightcyan5}{rgb}{0.27, 0.31, 0.31} %  69,  79,  79
% Reserved words' color (in programs)
\definecolor{codecolor}{rgb}{0,0,0.3}
% Other colors
\definecolor{green}{rgb}{0,.5,0}
\definecolor{lightgray}{gray}{0.94}
\definecolor{lightyellow}{rgb}{1, 0.99 , 0.92} 
\definecolor{lightyellow1}{rgb}{0.9, 0.89 , 0.82}
\definecolor{gray97}{gray}{.97}
\definecolor{gray75}{gray}{.75}
\definecolor{gray45}{gray}{.45}
%======================================================================

%======================================================================
% Source code stuff
%----------------------------------------------------------------------
\usepackage{listings}
\lstloadlanguages{C}      % C language
\lstloadlanguages{TeX}    % Tex language (for text)
\lstset{ %
  frame=Ltb,
  framerule=0pt,
  framextopmargin=3pt,
  framexbottommargin=3pt,
  framexleftmargin=3pt,
  framesep=0pt,
  rulesep=.4pt,
  backgroundcolor=\color{gray97},
  rulesepcolor=\color{black},
  % 
  stringstyle=\ttfamily,
  showstringspaces = true,
  %
  basicstyle=\ttfamily,
  commentstyle=\color{gray45},
  keywordstyle=\color{codecolor}\bfseries,
  % 
  numbers=left,
  numbersep=7pt,
  numberstyle=\tiny,
  numberfirstline = false,
  % 
  belowcaptionskip = 0pt,
  %
  literate=
  {á}{{\'a}}1
  {é}{{\'e}}1
  {í}{{\'i}}1
  {ó}{{\'o}}1
  {ú}{{\'u}}1
  {ñ}{{\~n}}1
  {Á}{{\'A}}1
  {É}{{\'E}}1
  {Í}{{\'I}}1
  {Ó}{{\'O}}1
  {Ú}{{\'U}}1
  {Ñ}{{\~N}}1
  {¡}{{!`}}1
  {¿}{{?`}}1
  {«}{{\guillemotleft{}}}1
  {»}{{\guillemotright{}}}1
}
%======================================================================

%======================================================================
% Rail alias configuration
%----------------------------------------------------------------------
\railalias{CRNL}{\char"5Cr\char"5Cn}
\railterm{CRNL}
\railnontermfont{\ttfamily\upshape}
%======================================================================


\title{Borrador de una propuesta de \\
  Descripción de la comunicación entre ¿qarmsim? y ¿armsim?}

\author{Sergio Barrachina Mir y Germán Fabregat Llueca}

% \date{20 de septiembre de 1999}


\begin{document}

\maketitle{}

\tableofcontents

\section{Introducción}

El protocolo de comunicación mediante sockets descrito en este
documento tiene por objeto regular la comunicación entre el simulador
de ARM y la interfaz gráfica.

El simulador hará las veces de servidor en la comunicación. Por lo
tanto, deberá ponerse a la escucha en un determinado puerto y atender
las peticiones que le vaya enviado la interfaz gráfica, respondiendo
en caso necesario al simulador sobre el resultado de las acciones
llevadas a cabo.

La interfaz gráfica deberá conectarse al puerto en el que esté
escuchando el simulador, le enviará las peticiones y recogerá las
respuestas generadas por el simulador

¿Cómo determinar el puerto en el que se lleva a cabo la comunicación?
Puesto que el usuario ejecutará la interfaz gráfica, ésta puede
encargarse de ejecutar a su vez una instancia del simulador,
aprovechando para indicarle que debe ponerse a la escucha y en qué
puerto debe hacerlo.

Así pues, la interfaz gráfica deberá encargarse de encontrar algún
puerto disponible en un determinado rango y de lanzar el simulador
indicándole en qué puerto debe ponerse a la escucha.

El simulador, a su vez, deberá proporcionar un parámetro de ejecución
«\verb+-p PORT+», que utilizará para saber que debe ponerse a la
escucha en el puerto indicado.


\section{Peticiones de la interfaz al simulador}

Una vez ejecutado el simulador, la interfaz abrirá una conexión con el
puerto prefijado. Dicha conexión únicamente se cerrará cuando el
usuario decida terminar la ejecución de la interfaz gráfica. La
interfaz gráfica enviará un último comando, «\texttt{EXIT}», al
simulador para indicarle que el usuario ha decidido cerrar el
programa. El simulador, a su vez, debería finalizar su ejecución en
cuanto reciba dicho comando de finalización, no sin antes liberar el
puerto en el que estuviera escuchando.

Cada una de las peticiones de la interfaz al simulador consistirá en
una línea de texto finalizada con los caracteres «\verb|\r\n|»,
emulando el funcionamiento de una conexión al simulador mediante el
comando «\verb+telnet+». (Es decir, se podrá utilizar el comando
«\verb|telnet|» para gobernar al simulador.)

La información intercambiada deberá estar codificada en
\texttt{UTF-8}.

La Figura~\ref{fig:ordenes} muestra las ordenes que deberá ser capaz
de atender el simulador.

\begin{figure}[htbp]
  \centering
  \begin{rail}
  (
    'SHOW' ('VERSION' | 'REGISTER' REGNAME
                      | 'MEMORY' ('BYTE' | 'HALF' | 'WORD') 'AT' ADDRESS 
                      | 'BREAKPOINTS' ) |
    'DUMP' ('REGISTERS' | 'MEMORY' ADDRESS NBYTES) |
    'DISASSEMBLE' ADDRESS NINSTS |
    'RESET' ('REGISTERS' | 'MEMORY' ) |
    'CLEAR' ('BREAKPOINTS' | 'BREAKPOINT' 'AT' ADDRESS) |
    'SET'  (
             ('REGISTER' REGNAME |
              'MEMORY' ('BYTE' | 'HALF' | 'WORD') 'AT' ADDRESS ) 'WITH' HEXVALUE | 
             ('BREAKPOINT' 'AT' ADDRESS)
           ) |
    'EXECUTE' ('ALL' | 'STEP' | 'SUBROUTINE') |
    'ASSEMBLE' PATH
  ) CRNL
  \end{rail}
  \caption{Sintaxis de los comandos soportados por el simulador}
  \label{fig:ordenes}
\end{figure}

\section{Respuesta del simulador}

La respuesta del simulador a la interfaz gráfica también se realizará
por medio de líneas de texto terminadas con los caracteres
«\verb|\r\n|».

Las respuestas del simulador se pueden clasificar en tres tipos en
función de la información que deba devolver a la interfaz gráfica:
\begin{description}
\item[Respuesta simple.] El simulador recibe la orden, ejecuta la
  acción y tan solo debe devolver una respuesta para indicar que ha
  completado la acción. Esta respuesta constará de una única línea con
  la palabra «\texttt{OK}».
\item[Respuesta con un número de líneas determinado.] El simulador
  debe proporcionar una respuesta relacionada con la acción llevada a
  cabo y ésta consta de un número de líneas conocido de antemano.
\item[Respuesta con un número de líneas variable.] El simulador debe
  proporcionar una respuesta, pero el número de líneas de la respuesta
  depende del resultado de la acción llevada a cabo.
\end{description}

Para los dos primeros casos no es necesario hacer nada especial. El
simulador ejecuta la acción y dependiendo del tipo en el que se
enmarque la acción, devolverá una única línea con el texto
«\texttt{OK}», o devolverá un número de líneas prefijado. La interfaz
gráfica, a su vez, ya sabe qué debe esperar en cada uno de estos
casos.

Para las respuestas que pertenezcan al tercer tipo, se debe enviar una
última línea con el texto «\verb+EOF+» para que la interfaz gráfica
sepa que no debe esperar más líneas de respuesta.

Las respuestas del simulador también pueden clasificarse según
dependan o no del resultado de una acción. Así, habrán acciones que se
llevarán a cabo sin problemas, p.e., mostrar la versión del simulador,
y acciones que por el contrario sí que podrían fallar, p.e., ensamblar
un determinado programa. Las acciones que no pueden fallar tienen una
única respuesta, mientras que las acciones que pueden fallar pueden
proporcionar dos respuestas distintas dependiendo del resultado de la
acción.

\section{Descripción de los comandos soportados y su respuesta}

\subsection{Comandos «\texttt{SHOW}»}

\subsubsection{SHOW VERSION}

El comando «\texttt{SHOW VERSION}» muestra información sobre la
versión del simulador. La interfaz gráfica mostrará la respuesta a
dicho comando en la ventana de mensajes. También se mostrará en el
cuadro de diálogo correspondiente a «\verb|Acerca de...|».


La respuesta al comando «\texttt{SHOW VERSION}» estará formada por un
número de líneas no determinado, por lo que deberá indicarse que se ha
concluido la respuesta por medio de una línea con el texto
«\texttt{EOF}». La gramática de la respuesta será por tanto:

\begin{rail}
 ( (TEXT CRNL) + () ) "EOF" CRNL
\end{rail}

A continuación se muestra un ejemplo de posible respuesta:
\begin{lstlisting}
ARMSIM version 1.0
(c) 2014 Germán Fabregat Llueca
EOF
\end{lstlisting}


\subsubsection{SHOW REGISTER REGNAME}

El comando «\texttt{SHOW REGISTER REGNAME}», donde «\texttt{REGNAME}»
es el nombre del registro que se quiere consultar, «\texttt{r0}» a
«\texttt{r15}», etc. Se utiliza para consultar el contenido en
hexadecimal del registro indicado. La respuesta a dicho comando será
una única línea con el nombre del registro consultado seguido del
contenido en hexadecimal de dicho registro (separados por «: »). Por
ejemplo:
\begin{lstlisting}
r3: 0x0001234
\end{lstlisting}


\subsubsection{SHOW MEMORY (BYTE|HALF|WORD) AT ADDRESS}

El comando «\texttt{SHOW MEMORY (BYTE|HALF|WORD) AT ADDRESS}», donde
«\texttt{ADDRESS}» es una posición de memoria expresada en
hexadecimal, sirve para consultar el contenido de la posición de
memoria indicada. La respuesta a dicho comando será una única línea
con la dirección de memoria consultada y con el contenido en
hexadecimal de dicha posición de memoria (separados por «:
»). Naturalmente, dependiendo del tipo de dato indicado se devolverá
el byte, la media palabra, o la palabra que se encuentre en dicha
posición de memoria.

Por ejemplo, la respuesta a «\texttt{SHOW MEMORY HALF AT 0x10010004}»
sería tal que así:
\begin{lstlisting}
0x1001004: 0x1234
\end{lstlisting}

\subsubsection{SHOW BREAKPOINTS}

El comando «\texttt{SHOW BREAKPOINTS}» se utiliza para consultar en
qué direcciones de memoria se han establecido puntos de ruptura. La
respuesta a dicho comando constará de tantas líneas como puntos de
ruptura, seguidas de una última línea con el texto «\texttt{EOF}». Por
ejemplo:
\begin{lstlisting}
0x00400024
0x00400100
0x00400104
0x00400106
EOF
\end{lstlisting}

\subsection{Comandos «\texttt{DUMP}»}

\subsubsection{DUMP REGISTERS}

El comando «\texttt{DUMP REGISTERS}» proporciona el volcado del
contenido de todos los registros del procesador. La respuesta
presentará la siguiente forma (con tantas líneas como registros
soportados):

\begin{rail}
  (REGNAME ': ' HEXVALUE CRNL) +
\end{rail}

Donde la primera línea mostrará el nombre contenido en hexadecimal del
registro \texttt{r0} (separados por «\texttt{: }»), la segunda línea,
el nombre y contenido del registro \texttt{r1}, y así
sucesivamente. Por ejemplo:

\begin{lstlisting}
r1: 0x00000000
r2: 0x12345678
\end{lstlisting}

\hspace{6ex}$\vdots$

\lstset{firstnumber=14}
\begin{lstlisting}
r14: 0x24000000
r15: 0x0040003E
\end{lstlisting}
\lstset{firstnumber=1}

\subsubsection{DUMP MEMORY ADDRESS NBYTES}

El comando «\texttt{DUMP MEMORY ADDRESS NBYTES}» proporciona el
volcado del contenido de memoria comenzando en la dirección
«\texttt{ADDRESS}» con un tamaño de «\texttt{NBYTES}». Cada byte se
mostrará en hexadecimal en una línea separada, por lo que la respuesta
constará de «\texttt{NBYTES}» líneas.

Es responsabilidad de la interfaz gráfica asegurarse de que todas las
direcciones de memoria que vayan a proporcionarse estén dentro del
rango permitido de direcciones de memoria.

\subsection{Comando «\texttt{DISASSEMBLE}»}

\subsubsection{DISASSEMBLE ADDRESS NINSTS}
\label{sec:disassemble}

El comando «\texttt{DISASSEMBLE ADDRESS NINSTS}» proporciona el
desensamblado de «\texttt{N}» instrucciones comenzando a partir de la
que se encuentra en la dirección «\texttt{ADDRESS}».

La respuesta constará de tantas líneas como instrucciones se hayan
indicado en el parámetro «\texttt{NINSTS}». Es responsabilidad de la
interfaz gráfica asegurarse de que el rango de instrucciones a
desensamblar está contenido en el rango de direcciones de memoria
posibles.

Cada línea de la respuesta tendrá la sintaxis mostrada a
continuación. En el caso de que una posición de memoria pueda ser
correctamente desensamblada, se devolverá el camino~1. En el caso de
que no pueda ser desensamblada, el camino~2.

\begin{rail}
 '[' ADDR ']' HEX16
  (
    [1]  HEX16? MACHINE ( () | ';' N ASM) | 
    [2] 'NOT AN INSTRUCTION'
  ) CRNL
\end{rail}

Donde:
\begin{description}
\item[ADDR] Es la dirección de memoria desensamblada.
\item[HEX16] Es la instrucción codificada en hexadecimal con
  16~bits. Puesto que la longitud de las instrucciones es variable,
  aquellas que ocupen 32~bits proporcionarán los segundos 16~bits de
  la instrucción en otro número hexadecimal.
\item[MACHINE] Es el desensamblado de la instrucción.
\item[N] Es el número de línea en el código fuente que ha dado pie a
  esta instrucción.
\item[ASM] Es la instrucción en el código fuente.
\end{description}

\subsection{Comandos «\texttt{RESET}»}

\subsubsection{RESET REGISTERS}

El comando «\texttt{RESET REGISTERS}» se utiliza para restituir a
todos los registros a su estado inicial.

Como respuesta se proporciona una única línea con el texto
«\texttt{OK}».

\subsubsection{RESET MEMORY}

El comando «\texttt{RESET MEMORY}» se utiliza para restaurar a sus
valores iniciales toda la memoria del simulador.

Como respuesta se proporciona una única línea con el texto
«\texttt{OK}».


\subsection{Comandos «\texttt{CLEAR}»}

\subsubsection{CLEAR BREAKPOINTS}

El comando «\texttt{CLEAR BREAKPOINTS}» se utiliza para borrar todos
los puntos de ruptura fijados anteriormente.

Como respuesta se proporciona una única línea con el texto
«\texttt{OK}».

\subsubsection{CLEAR BREAKPOINT AT ADDRESS}

El comando «\texttt{CLEAR BREAKPOINT AT ADDRESS}» se utiliza para
eliminar el punto de ruptura fijado en la dirección
«\texttt{ADDRESS}».

Como respuesta se proporciona una única línea con el texto
«\texttt{OK}».

Nota: el simulador contesta siempre con «\texttt{OK}», aunque el punto
de ruptura especificado no estuviera previamente fijado.

\subsection{Comandos «\texttt{SET}»}

\subsubsection{SET REGISTER REGNAME WITH HEXVALUE}

El comando «\texttt{SET REGISTER REGNAME WITH HEXVALUE}», donde
«\texttt{REGNAME}» es el nombre de uno de los registros soportados por
el simulador, sirve para sobreescribir el contenido del registro
«\texttt{REGNAME}» con el valor en hexadecimal indicado por el
parámetro «\texttt{HEXVALUE}».

Como respuesta se proporciona una única línea con el texto
«\texttt{OK}».

\subsubsection{SET MEMORY (BYTE|HALF|WORD) AT ADDRESS WITH HEXVALUE}

El comando «\texttt{SET MEMORY (BYTE|HALF|WORD) AT ADDRESS WITH
  HEXVALUE}», sirve para sobreescribir el contenido de la posición de
memoria «\texttt{ADDRESS}» con el valor en hexadecimal indicado por el
parámetro «HEXVALUE». Naturalmente, «\texttt{HEXVALUE}» será un número
hexadecimal de 8, 16, o 32~bits, dependiendo de si se ha especificado
«\texttt{BYTE}», «\texttt{HALF}» o «\texttt{WORD}» en el comando,
respectivamente.

Como respuesta se proporciona una única línea con el texto
«\texttt{OK}».

\subsubsection{SET BREAKPOINT AT ADDRESS}

El comando «\texttt{SET BREAKPOINT AT ADDRESS}», sirve para fijar un
punto de ruptura en la posición de memoria «\texttt{ADDRESS}».

Como respuesta se proporciona una única línea con el texto
«\texttt{OK}».


\subsection{Comando «\texttt{EXECUTE}»}

\subsubsection{EXECUTE (ALL|STEP|SUBROUTINE)}

El comando «\texttt{EXECUTE (ALL|STEP|SUBROUTINE)}» se utiliza para
indicar al simulador que ejecute una secuencia de instrucciones
empezando por la referenciada por el registro «\texttt{PC}».

La ejecución de una instrucción puede detenerse por alguna de las
siguientes condiciones de parada:
\begin{itemize}
\item El valor leído de memoria no pueda desensamblarse correctamente.
\item Se alcance un punto de ruptura.
\item Se alcance el final del programa.
\end{itemize}

El funcionamiento del comando «\texttt{EXECUTE}» depende del término
que le acompañe. Para cada uno de los términos posibles, el
funcionamiento es el siguiente:
\begin{description}
\item[\texttt{ALL}] Se ejecutan todas las instrucciones posibles hasta
  que se alcance cualquiera de las condiciones de parada descritas
  anteriormente.
\item[\texttt{STEP}] Tan solo se ejecuta una instrucción.
\item[\texttt{SUBROUTINE}] Si la instrucción en curso es una llamada a
  una subrutina, se ejecutan todas las instrucciones de la subrutina,
  salvo que se encuentre alguna de las condiciones de parada descritas
  anteriormente.
\end{description}

Una vez ejecutadas las instrucciones que correspondan, se devolverá
una respuesta formada por las siguientes partes:

\begin{rail}
  RESULTCODE CRNL DISASSEMBLE CRNL STATUSLINES
\end{rail}

La primera línea de la respuesta mostrará uno de los siguientes
términos:
\begin{itemize}
\item «\texttt{SUCCESS}», si se ha podido ejecutar dicha instrucción;
\item «\texttt{ERROR}», si se ha producido un error al intentar
  ejecutar dicha instrucción;
\item «\texttt{BREAKPOINT REACHED}», si se ha alcanzado un punto de
  ruptura; o
\item «\texttt{END OF PROGRAM}», si se ha alcanzado el final del
  programa.
\end{itemize}

La segunda línea de la respuesta mostrará el desensamblado de la
última instrucción que se ha intentado ejecutar con el mismo formato
que el devuelto por el comando «\texttt{DISASSEMBLE}». La descripción
detallada de dicha sintaxis se puede consultar en el
Apartado~\ref{sec:disassemble}. A modo de recordatorio, ésta era:

\begin{rail}
 '[' ADDR ']' HEX16
  (
    [1]  HEX16? MACHINE ( () | ';' N ASM) | 
    [2] 'NOT AN INSTRUCTION'
  ) CRNL
\end{rail}

Las siguientes líneas mostrarán los registros afectados, si los
hubiera, las posiciones de memoria afectadas, si las hubiera, y el
motivo del error, en el caso de que se haya producido uno.

En cuanto a los registros y posiciones de memoria afectadas, puesto
que el objetivo es el de mostrar gráficamente qué registros y
posiciones de memoria se han visto afectadas, cualquier registro o
posición de memoria afectado deberá incluirse en la lista devuelta a
la interfaz gráfica, aunque el nuevo valor sea el mismo que el que ya
tenía.

Por otro lado, en el caso de que la instrucción no hubiera podido
ejecutarse, la información sobre el motivo del error será simplemente
una secuencia de líneas de texto, con una última línea con el texto
«\texttt{EOF}». Hay que tener en cuenta que la información que se
proporcione sobre el error se reproducirá tal cual en la ventana de
mensajes de la interfaz gráfica, por lo que en el caso de que no se
pueda ejecutar una determinada instrucción debería proporcionarse
información lo más completa posible sobre el motivo por el que no se
ha podido ejecutar dicha instrucción.

La sintaxis detallada de la parte final de la respuesta, en la que se
muestran los registros y posiciones de memoria afectados, así como en
su caso, el motivo del error, es la siguiente:
\begin{rail}
StatusLines:
('AFFECTED REGISTERS' CRNL ( ( NAME ': ' HEX32 CRNL ) + () ) ) ? \\
('AFFECTED MEMORY' CRNL ( ( ADDR ': ' HEX8 CRNL ) + () ) ) ? \\
('ERROR MESSAGE' CRNL ( ( ERRORLINE CRNL ) + () ) ) ? \\
'EOF' CRNL
\end{rail}



\subsection{Comando «\texttt{ASSEMBLE}»}

\subsubsection{ASSEMBLE PATH}

El comando «\texttt{ASSEMBLE PATH}» sirve para indicarle al simulador
que debe compilar el fichero que se encuentra en «\texttt{PATH}».

POR DETERMINAR.

\end{document}

%%% Local Variables: 
%%% mode: latex
%%% TeX-master: t
%%% End: 
